\documentclass {article}
\usepackage{amsmath}
\begin {document}

\title {Problem Set 5 Writeup}
\author {Joshua Cole}
\date {Dec 14, 2013}
\maketitle
\section {Exercises on transposes, permutations, spaces}

\paragraph{Problem 5.1:}

a) Find a 3 by 3 permutation matrix with $P^4 = I$ (but not $P = I$). b) Find a 4 by 4 permutation $P$ with $P^4= I$

$$
\begin{bmatrix}
0 & 0 & 1 \\
1 & 0 & 0 \\
0 & 1 & 0
\end{bmatrix}
\begin{bmatrix}
1 & 0 & 0 & 0 \\
0 & 0 & 0 & 1 \\
0 & 1 & 0 & 0 \\
0 & 0 & 1 & 0
\end{bmatrix}
$$


\paragraph{Problem 5.2:} Suppose A is a four-by-four matrix. How many entries can be chosen indepently if A is symmetric? How about if A is skew-symmetric?

Ten entries can be chosen if $A$ is symmetric and 6 entries can be chosen if $A$ is skew symmetric.

\paragraph{Problem 5.3:} True or false (check addition or give a counterex­ample):

a) The symmetric matrices in $M$ (with $A^T = A$) form a subspace.

b) The skew-symmetric matrices in $M$ (with $A^T = −A$) form a subspace.

c) The unsymmetric matrices in $M$ (with $A^T \ne A$) form a subspace.

a) True. b) True. c) False due to 
$$
\begin{bmatrix}
1 & 1 \\
0 & 0 
\end{bmatrix} +
\begin{bmatrix}
0 & 0 \\
1 & 1 
\end{bmatrix}
=
\begin{bmatrix}
1 & 1 \\
1 & 1 
\end{bmatrix}
$$
\end {document}